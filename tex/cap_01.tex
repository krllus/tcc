\chapter{Introdu��o}
\label{cap:intro}
%contexto
	O desenvolvimento de aplica��es mobile e web, tem se tornado cada vez mais popular, com diversos aplicativos surgindo no mercado diariamente. Com o mercado cada vez mais competitivo, para que uma aplica��o continue sendo usada, ela deve se manter �til ao usu�rio. Uma forma de ser �til � sempre apresentar o conte�do que usu�rio deseja de forma r�pida.     
      
%problema
	De forma a atender as exig�ncias dos usu�rios, se faz necess�rio a utiliza��o de uma infraestrutura que cubra essas necessidades da aplica��o. Porem, para pequenas empresas e desenvolvedores \textit{solo}, criar e manter servidores � custoso quanto a tempo e m�o de obra.

%solu��o
	Com o objetivo de facilitar o desenvolvimento de aplica��es mobile e web, surgem empresas que fornecem \textit{Backend} como um servi�o de nuvem. Eliminando a necessidade de cria��o de uma infraestrutura local, deixando desenvolvedores livre para focar em seus aplicativos.

%avalia��o

%resumo outros capitulos
	Este trabalho est� organizado da seguinte forma: o Cap�tulo \ref{cap:fund_teorica} apresenta a fundamenta��o te�rica, abordando os conceitos relacionados a computa��o em nuvem e seus provedores de servi�o; o Cap�tulo \ref{cap:challenges} tr�s os principais desafios envolvidos no uso de servi�os de \textit{Backend} hospedados em ambiente de computa��o nuvem; o Cap�tulo \ref{cap:minha_ufg} apresenta um estudo de caso por meio da implementa��o do aplicativo Minha UFG; e o Cap�tulo \ref{cap:conclusion} apresenta as considera��es finais.

