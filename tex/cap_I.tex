\chapter{Back-end como serviço (BaaS)}

	%contexto explicar o que seriam softwares como servico (SaaS)

	Em tempos não muito distantes ou até atuais, empresas hospedavam seus proprios servidores de infraestrutura que são responsaveis por todo o back-end, assim ficando responsavel pelos desafios relacionados a gerencia de TI tais como disponibilidade, escalabilidade e manutenção por exemplo. Esse cenario vem se modificando ao longo dos ultimos anos. Empresas estão migrando para o modelo de Software as a Service(SaaS), aonde serviços de infraestrutura são fornecidos por outra empresa, passando para o fornecedor a responsabilidade de manutenção do serviço. 
	Agilização no desenvolvimento de novas soluções, menos custo inicial, integridade e consistencia dos dados são as principais vantagens do modelo de SaaS.
		
	\noindent Pro:
	\begin{itemize}		
		\item servidor já está pronto, somente configurar pequenas coisas e já está funcionando.
		\item menor custo inicial para criar uma solução.
		\item segurança, autorização, autenticação, escalabilidade, disponibilidade e manutenção passam a ser terceirizados pela empresa contratada
	\end{itemize}

	\noindent Con: 
	\begin{itemize}

		\item tem que pagar para utilizar, tazas variam de acordo com os serviços utilizados
		\item mudar de uma arquitetura para outra as vezes pode ser trabalhoso.
	\end{itemize}

\section{Definição}
	
	Back-end as a service (Back-end como serviço) é um tipo de SaaS que prevê o suporte para que uma ou mais aplicações funcionem sem a necessidade do desenvolvimento back-end. Este tipo de serviço, no geral, provê armazenamento 


	Este trabalho tem como objetivo a analise e discussão das funcionalidades e desafios referentes ao serviço Firebase. Visto ques este é considerado o mais utilizado da atualidade. \ref{firebase_most_used_service}

\section{Funcionalidades / Oportunidades}
\label{sec:functionality}
	As funcionalidades dos Back-ends como serviços variam de acordo com a empresa que está provendo o serviço e com qual finalidade o serviço foi contratado, dentre os serviços mais comuns estão armazenamento de dados, gerenciamento de usuários, notificações assincronas e integração com redes sociais (ref 11 e 12 texto franciely)

	\subsection{Gerenciamento de Usuários} % AKA autenticação
	\label{subsec:user_management}
	% definição
	em geral o que é: disponibiliza ao desenvolvedor todos os meios para o controle de acesso de usuários a sua aplicação;
	
	% contexto
	Todo sistema possui usuarios cadastrados e as devidas permissões que cada um deles tem ao sistema. BaaS oferecem serviços diferentes de acordo com as necessidades do cliente. Firebase por exemplo, fornece o Firebase Authentication, que permite login anonimo, via email e senha ou usando provedores populares de identidade federada tais como o Google, Facebook, Twitter entre outros. Firebase Authentication utiliza os padrões da industria tais como OAuth 2.0 and OpenID Connect, facilitando a integração com um backend custom caso necessario.


	Um usuario logado no seu app é representado por um objeto do tipo Firebase User. Essa  classe possui um conjunto de propriedades basicas - um ID unico, um email principal, um nome e uma URL de uma foto - que são armazenados no banco de dados da aplicação. Caso seja necessario, informações adicionais podem ser armazenadas dentro do Firebase Realtime Database. Firebase User mantem informações dos diferentes provedores usados para autenticar, permitindo que seja possivel atualizar propriedades que faltavam, usando informações de outros provedores.

	\subsection{Gerenciamento de Dados}
	\label{subsec:data_management}
	% definição
	o que é: disponibiliza ao desenvolvedor meios para armazenar e acessar os dados gerados por sua aplicação. No geral, a gerencia de dados é feita através de interfaces simples e intuitivas (ref 11 - francielly);
	
	% contexto
	Storage do firebase, armazenamento de dados criados pelos usuarios do sistema

	\subsection{Gestão de Permissão} % aka autorização
	\label{subsec:user_authorization}
	
	A funcionalidade de banco de dados como serviço é uma das mais utilizadas dentro dos BaaS. Dessa maneira, por questões de manutenção de integridade, privacidade e segurança, os sistemas do mercado oferecem diferentes maneiras para a a definição de regras de acesso e validação. O objetivo é deixar livre a tentativa de persistência de informação para todos os usuários e com qualquer tipo de dado. Porém, tudo o que for submetido a gravação passará por um controle restrições de acesso e limites internos.
	
	No Firebase, esse controle de segurança e regras é implementado usando um conjunto de regras definidas pelo programador através de descritores JSON. Estas são aplicadas a cada operação de leitura ou escrita, fazendo com que uma requisição seja valida somente se essas regras permitirem. \cite{https://firebase.google.com/docs/database/security/}. Veja um exemplo no trecho de código abaixo, onde todos os usuários podem ler o nó foo, mas ninguém pode lê-lo. 
	
	%exemplo%
	\begin{center}
 		\begin{minipage}{0.7\textwidth}
  			\begin{codigo}[H]
   				\small
   				\VerbatimInput[xleftmargin=10mm,numbers=left,obeytabs=true]{./prog/rules_example.json}
   				\caption{\texttt{Exmemplo de Regras} }
   				\label{code:rules_example}
  			\end{codigo}
 		\end{minipage}
	\end{center}
	
	\subsection{Objetos Customizaveis}
	% definição

	% contexto

	\subsection{API}
	% definição

	% contexto

	\subsection{Disponibilidade e Escalabilidade}
	% definição

	% contexto

	\subsection{Integração com redes sociais}
	% definição

	% contexto

\section{Precificação}%6:30
\label{sec:pricing}
	% precificação gera desafios.
	% definição

	% contexto
	O backend como serviço, assim como todo software que roda como serviço na nuvem, ele é mantido por uma empresa que obviamente vai cobrar pelo serviço, por tanto, deve haver uma preocupação com o valor a ser pago para o fornecedor dessa solução.
	
	O preço varia muito de acordo com a quantidade do uso do serviço, que depende do conjunto de varios fatores, dentre ele podemos citar, por exemplo no firebase estes:
		ponto 1 paga por quantidade de dados consumidos
		ponto 2 paga pela quantidade de usuarios simultaneos
		...

	% falar dos planos

\section{Desafios}
	% definição
	não dar ideia de bug... nao vai gerar problemas, vai gerar desafios.. questoes de projetos que levam aos desafios da escolha
	
	% contexto

	\subsection{Escalabilidade}%7:30
	% definição
	o que é escalabilidade

	% contexto
	quando estamos tratando de um baas, assim como um backend que é gerido pelo usuario dentro de um servidor local ou servidor de uma maquina virtual contratada, voce tambem deve verificar a capacidade de escalabilidade do mesmo.

	Cada fornecedor tem as capacidades minimas e maximas possivel, por exemplo 	
	servidor um... max min...

	\subsection{Flexibilidade}%8:55 {antigo 'mudança de provedor'}
	% definição

	% contexto
	dentro da manutenção de software uma das ações que podem ser realizada é a troca de fornecedor de algum serviço, por exemplo a troca do backend como serviço utilizado.

	dependendo da estrutura utilizada para armazenar os dados no serviço, trocar de provedor pode ser uma tarefa complicada, pois nao necessariamente os dados serao armazenados da mesma forma, algumas APIs disponibilizadas por um podem nao ter a mesma funcionalidades que o do outro... toda a parte que trata da comunicação entre o servidor e a criação dos objetos tera de ser repensada.

	% soluções utilizadas

	\subsection{Segurança}%9:30
	% 
	% definição
	um backend deve oferecer segurança no serviço que ele oferece, seja ele um serviço ou gerido por um servidor ques esta instalado num predio da organozação.

	os baas em geral tratam daquela maneira
		utilizam um canal seguro via tts para transmissao de informaçoes
		e portanto o cliente deve estar praparado para esse tipo de comunicação

	ponto importante a destacar
	em relação aos dados em geral não temos muita informação, isso deve ser implementado pelo desenvolvedor. portanto boa parte da logica de segurança fica implementada na aplicação cliente. deve se ter um cuidado de como é feita a implementação para não expor a todo o sistema pois o codigo está no cliente.

	% contexto

	% soluções utilizadas

	\subsection{Sincronização de dados}%14:40
	% tratar garantia de integridade junto
	% definição
	é uma ação que é muito importe entre o sistema que trata de diferentes versoes da informaçao em diferentes clientes. esse tipo de atividade geralmente gera um problema arquitetural que envolve tanto o momento que vai ser sincronizado e o que vai ser sincronizado e como vai ser sincronizado.

	alguns serviços fazem uma camada de transparencia...
	trafegam apenas as mudanças.

	% contexto
	
	% soluções utilizadas

	\subsection{Garantia da Integridade}%15:50
	% definição

	% contexto

	% soluções utilizadas

	\subsection{Garantia da consistencia da informação}%16:30
	% definição

	% contexto
	assim como explicado na seção \ref{sec:pricing}

	% soluções utilizadas

	\subsection{Minimizar trafego}
	% definição

	% contexto
	
	% soluções utilizadas

	\subsection{Minimizar espaço utilizado}
	% definição

	% contexto
	
	% soluções utilizadas

	\subsection{Recuperação de dados}%19:30
	% definição

	% contexto
	
	% soluções utilizadas
