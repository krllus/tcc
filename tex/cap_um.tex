\chapter{Back-end como servi�o}

	explicar o que seriam softwares como servico (SaaS)

\section{Defini��o}
	(Copia)
	Back-end as a service (Back-end como servi�o) � um tipo de SaaS que prev� o
suporte para que uma ou mais aplica��es funcionem sem a necessidade do
desenvolvimento back-end. 

\section{Funcionalidades}
	
	As funcionalidades dos Back-ends como servi�os varia de acordo com a empresa que est� provendo o servi�o e com qual finalidade o servi�o foi contratado, dentre os servi�os mais comuns est�o armazenamento de dados, gerenciamento de usu�rios, notifica��es assincronas e integra��o com redes sociais (ref 11 e 12 texto franciely)

	\subsection{Gerenciamento de Usu�rios}
	
	em geral o que �: disponibiliza ao desenvolvedor todos os meios para o controle de acesso de usu�rios a sua aplica��o;

	Todo sistema possui usuarios cadastrados e as devidas permiss�es que cada um deles tem ao sistema. BaaS oferecem servi�os diferentes de acordo com as necessidades do cliente. Firebase por exemplo, permite login anonimo ou usando redes sociais ou e-mail, tambem permite gerir quais as permiss�es que cada usuario tem.

	\subsection{Gerenciamento de Dados}
	o que �: disponibiliza ao desenvolvedor meios para armazenar e acessar os dados gerados por sua aplica��o. No geral, a gerencia de dados � feita atrav�s de interfaces simples e intuitivas (ref 11);

	Storage do firebase, armazenamento de dados criados pelos usuarios do sistema

	\subsection{Objetos Customizaveis}
	\subsection{API}
	\subsection{Disponibilidade e Escalabilidade}
	\subsection{Integra��o com redes sociais}


\section{Problemas}
	
	\subsection{Precifica��o}
	paga por quantidade de dados consumidos
	paga pela quantidade de usuarios simultaneos

	\subsection{Escalabilidade}
	a estrutura dos dados � importante para a escalabilidade do servi�o
	nem sempre 

	\subsection{Mudan�a de provedor}
	dependendo da estrutura utilizada para armazenar os dados no servi�o, trocar de provedor pode ser uma tarefa complicada, pois nao necessariamente os dados serao armazenados da mesma forma, algumas APIs disponibilizadas por um podem nao ter a mesma funcionalidades que o do outro... toda a parte que trata da comunica��o entre o servidor e a cria��o dos objetos tera de ser repensada.

	\subsection{Seguran�a}
	\subsection{Sincroniza��o de dados}
	\subsection{Garantia da Integridade}
	\subsection{Garantia da Consistencia da Informa��o}
	\subsection{Espa�o utilizado}