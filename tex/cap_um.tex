\chapter{Back-end como servi�o}

	baseado fortemente no texto da francielly

	explicar o que seriam softwares como servico (SaaS)

\section{O que �}
	(Copia)
	Back-end as a service (Back-end como servi�o) � um tipo de SaaS que prev� o
suporte para que uma ou mais aplica��es funcionem sem a necessidade do
desenvolvimento back-end. 

\section{Funcionalidades}

	Este tipo de servi�o, no geral, prov� armazenamento de dados e um conjunto de
funcionalidades que na maioria das vezes cont�m gerenciamento de usu�rios,
notifica��es ass�ncrona e integra��o com redes sociais (ref 12) (ref 11)

	\subsection{Gerenciamento de Usuarios} 
	disponibiliza ao desenvolvedor todos os meios para o controle de acesso de 
usu�rios a sua aplica��o;
	
	\subsection{Gerenciamento de Dados}
	disponibiliza ao desenvolvedor meios para armazenar e acessar os dados gerados
por sua aplica��o. No geral, a gerencia de dados � feita atrav�s de interfaces
simples e intuitivas (ref 11);

	\subsection{Aramazenamento de Imagens}
	\subsection{Objetos Customizaveis}
	\subsection{API}
	\subsection{Disponibilidade e Escalabilidade}
	\subsection{Integra��o com redes sociais}

\section{Problemas}
	
	\subsection{Precifica��o}
	paga por quantidade de dados consumidos
	paga pela quantidade de usuarios simultaneos

	\subsection{Escalabilidade}
	a estrutura dos dados � importante para a escalabilidade do servi�o
	nem sempre 

	\subsection{Mudan�a de provedor}
	dependendo da estrutura utilizada para armazenar os dados no servi�o, trocar de provedor pode ser uma tarefa complicada, pois nao necessariamente os dados serao armazenados da mesma forma, algumas APIs disponibilizadas por um podem nao ter a mesma funcionalidades que o do outro... toda a parte que trata da comunica��o entre o servidor e a cria��o dos objetos tera de ser repensada.

	\subsection{Seguran�a}
	\subsection{Sincroniza��o de dados}
	\subsection{Garantia da Integridade}
	\subsection{Garantia da Consistencia da Informa��o}
	\subsection{Espa�o utilizado}