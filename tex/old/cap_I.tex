\chapter{Back-end como serviço (BaaS)}

	%contexto explicar o que seriam softwares como servico (SaaS)

	Em tempos não muito distantes ou até atuais, empresas hospedavam seus proprios servidores de infraestrutura que são responsaveis por todo o back-end, assim ficando responsavel pelos desafios relacionados a gerencia de TI tais como disponibilidade, escalabilidade e manutenção por exemplo. Esse cenario vem se modificando ao longo dos ultimos anos. Empresas estão migrando para o modelo de Software as a Service(SaaS), aonde serviços de infraestrutura são fornecidos por outra empresa, passando para o fornecedor a responsabilidade de manutenção do serviço. 

	Agilização no desenvolvimento de novas soluções, menor custo inicial, integridade e consistencia dos dados são as principais vantagens do modelo de SaaS.
		
	% definir principais vantagens
	% referencia?
	\noindent Pro:
	\begin{itemize}		
		\item é necessario customizar apenas pequenos arquivos para que o serviço esteja funcionando.
		\item custo inicial para uma deteminada solução é menor.
		\item segurança, autorização, autenticação, escalabilidade, disponibilidade e manutenção passam a ser serviços terceirizados pela empresa contratada.
	\end{itemize}

	%definir principais desvantagens
	% referencia
	% nao deixar de incluir caso do firebase quanto a questão de busca no banco representado na query abaixo: 
	% SELECT * FROM places WHERE name LIKE ""
	\noindent Con: 
	\begin{itemize}

		\item tem que pagar para utilizar e o preço cobrado varia de acordo com a utilizaçao dos serviços.
		\item mudar de uma arquitetura para outra as vezes pode ser trabalhoso.

	\end{itemize}

\section{Definição}
	
	Back-end as a service (Back-end como serviço) é um tipo de SaaS que prevê o suporte para que uma ou mais aplicações funcionem sem a necessidade do desenvolvimento back-end. Este tipo de serviço, no geral, provê armazenamento 

	% acho que preciso definir SaaS, em lugar algum do texto eu cito o que são SaaS. talvez na introdução mas não tenho certeza se é aceitavel
	% quais os tipos basicos de serviços mais disponibilizados por BaaS?
	% referencia para tal informação é necessaria

	Este trabalho tem como objetivo a analise e discussão das funcionalidades e desafios referentes ao serviço Firebase. Visto que este é considerado o mais utilizado da atualidade. \ref{firebase_most_used_service}

	% é necessario a referencia pra afirmar isso =/

\section{Funcionalidades / Oportunidades}
\label{sec:functionality}
	As funcionalidades dos Back-ends como serviços variam de acordo com a empresa que está provendo o serviço e com qual finalidade o serviço foi contratado, dentre os serviços mais comuns estão armazenamento de dados, gerenciamento de usuários, notificações assincronas e integração com redes sociais (ref 11 e 12 texto franciely)

	% paragrafo que explica o porque de comecar a falar de novas 'features'?

	\subsection{Gerenciamento de Usuários} % AKA autenticação
	\label{subsec:user_management}
	% definição
	em geral o que é: disponibiliza ao desenvolvedor todos os meios para o controle de acesso de usuários a sua aplicação;
	
	% contexto
	Todo sistema possui usuarios cadastrados e as devidas permissões que cada um deles tem ao sistema. BaaS oferecem serviços diferentes de acordo com as necessidades do cliente. Firebase por exemplo, fornece o Firebase Authentication, que permite login anonimo, via email e senha ou usando provedores populares de identidade federada tais como o Google, Facebook, Twitter entre outros. Firebase Authentication utiliza os padrões da industria tais como OAuth 2.0 and OpenID Connect, facilitando a integração com um backend custom caso necessario.


	Um usuario logado no seu app é representado por um objeto do tipo Firebase User. Essa  classe possui um conjunto de propriedades basicas - ID unico, email principal, nome e a URL de uma foto - que são armazenados no banco de dados da aplicação. Caso seja necessario, informações adicionais podem ser armazenadas dentro do Firebase Realtime Database. Firebase User mantem informações dos diferentes provedores usados para autenticar, permitindo que seja possivel atualizar propriedades que faltavam, usando informações de outros provedores.

	% referencia do site deles
	% falta escrever sobre algo aqui?

	\subsection{Gerenciamento de Dados}
	\label{subsec:data_management}
	% definição
	o que é: disponibiliza ao desenvolvedor meios para armazenar e acessar os dados gerados por sua aplicação. No geral, a gerencia de dados é feita através de interfaces simples e intuitivas (ref 11 - francielly);
	
	% contexto
	Usuarios estão sempre produzindo e consumindo conteudos, que podem ser desde arquivos diversos como fotos, documentos, ou dados que podem ser do tipo chave valor, endereço - rua. Assim BaaS oferecem APIs que facilitam para o desenvolvedor o acesso e armazenamento desses dados. 


	No Firebase existem duas APIs que tratam de armazenamento de dados, Firebase Storage, que trata do armazenamento e arquivos e o Firebase Realtime Database que trata do armazenamento de conjuntos chave-valor.
	
	O Firebase Realtime Database, é um banco de dados NoSQL oferecido pela plataforma do Firebase, aonde todos os dados são armazenados no formato JSON e disponibilizados em tempo real para todos os clientes que estão conectados. A API trata de sincronização e armazenamento offline, então se ocorrer uma atulização nos dados e o cliente não estava conectado, ele ainda podera utilizar dos dados que estavam em cache e posteriormente, quando uma conexação for estabelecida, os dados serão sincronizados.
	\cite{https://firebase.google.com/docs/database/}.

	% resalva da limitação de 10 mb?

	O Firebase Storage, é a plataforma utilizada para armazenamento de arquivos. Ela é fortemente integrada com o Goole Cloud Platform, 
	\cite{https://firebase.google.com/docs/storage/}.

	% finalizar esse paragrafo acima.

	\subsection{Gestão de Permissão} % aka autorização
	\label{subsec:user_authorization}
	A funcionalidade de banco de dados como serviço é uma das mais utilizadas dentro dos BaaS [referencia para essa afirmacao]. Dessa maneira, por questões de manutenção de integridade, privacidade e segurança, os sistemas do mercado oferecem diferentes maneiras para a a definição de regras de acesso e validação. O objetivo é deixar livre a tentativa de persistência de informação para todos os usuários e com qualquer tipo de dado. Porém, tudo o que for submetido a gravação passará por um controle restrições de acesso e limites internos.
	
	No Firebase, esse controle de segurança e regras é implementado usando um conjunto de regras definidas pelo programador através de descritores JSON. Estas são aplicadas a cada operação de leitura ou escrita, fazendo com que uma requisição seja valida somente se essas regras permitirem. \cite{https://firebase.google.com/docs/database/security/}. 
	
	Veja um exemplo no trecho de código abaixo, onde todos os usuários podem ler o nó foo, mas ninguém pode escrever nele. 
	
	%exemplo
	\begin{center}
 		\begin{minipage}{0.7\textwidth}
  			\begin{codigo}[H]
   				\small
   				\VerbatimInput[xleftmargin=10mm,numbers=left,obeytabs=true]{./prog/rules_example.json}
   				\caption{\texttt{Exmemplo de Regras} }
   				\label{code:rules_example}
  			\end{codigo}
 		\end{minipage}
	\end{center}
	
	\subsection{Objetos Customizaveis}
	\label{subsec:custom_objects}
 	Objetos do mundo real são complexos, e em muitas das vezes somente os tipos primitivos de dados não são suficientes para representar os atributos desses objetos. Assim necessitando que DAOs (Data Access Object) sejam criados justamente para converter essa estrutura complexa em uma que o banco de dados consegue entender.

 	No Firebase, os dados são armazenados como JSON assim, a lista de tipos primitiva é composta de tipos que podem ser convertidos para JSON diretamente, contendo String, Long, Double, Map<String,Object> e List<Object> isso falando de Java. Para usar objetos personalidados, a classe que o implementa deve de utilizar ter um construtor vazio e metodos publicos de acesso para os atributos dessa classe. \cite{https://firebase.google.com/docs/database/android/read-and-write}

	\subsection{API}
	\label{sub:api}
	% Definição
	possibilidade de o desenvolvedor implementar sua propria api sobre o serviço contratado, extendendo e/ou personalizando o funcionamento de uma funcionalidade ja existente.

	% contexto
	em alguns casos, é necessario implementar logicas de negocio mais robustas, que exigem mais processamento do dispositivo do cliente, ou precisam ser ocultadas por questoes de segurança. Exemplo seria a geração da thumbnail de imagens ou {arrumar outro exemplo relacionado a segurança}

	% arrumar um exemplo que trate da seguranca da informacao que deve ser realizada somente server-side

	Para resolver esse problema, o Firebase desenvolveu o Cloud Functions, que permite a criação de um serviço hospedado por eles onde operações custosas quanto ao processamento ou que não deveriam ser feitas no cliente, são realizadas nos servidores do proprio firebase.

	% melhor explicar o que seria o firebase cloud functions
	% exemplo pratico com codigo de thumbnail

	\subsection{Escalabilidade e Disponibilidade}
	\label{subsec:scalability_and_availability}
	% contexto
	Quando se tem um servidor interno, uma das principais preocupações é com a escalabilidade e disponibilidade. Ou seja como que o seu serviço vai funcionar quando esse servidor estiver sobre grande carga, que pode ser gerada por varios usuarios ativos ao mesmo tempo, varias operações de acesso ao banco de dados ocorrendo ao mesmo tempo ou até mesmo falta de recursos. Um sistema é dito escalavel se permanece eficiente quando há um aumento significativo no número de recursos e no numero de usuarios.\cite {colouris: p31 escalabilidade}. A disponibilidade de um sistema é medida da proporção de tempo em que ele está pronto para uso \cite {colouris: p33 tratamento de erro}

	BaaS investem em infraestrutura para mantem o serviço escalavel e disponivel na medida que a demanda dos usuarios vai crescendo, deixando o programador responsavel por estruturar os dados e as consultas de forma a minimizar a quantidade requisições e quantidade de nós a serem acessadas para realizar uma consulta. 

	% como firebase resolve problemas de escalabilidade 
	No Firebase, os problemas de usuarios simultaneos e escalabilidade são resolvidos pela plataforma, porem existe um custo de acordo com o que o cliente precisa que deve ser acertado entre as partes contratantes do serviço e o fornecedor. 

	% sao resolvidos pela plataforma, nao sei explicar como... devo entrar com os numeros dos servidores?
	% devo abordar o problema do custo do servico?

\section{Precificação}%6:30
	% o que nos leva a precifica;ao, um das principais vantagens e tambem desvantagens
\label{sec:pricing}
	% precificação gera desafios.
	% definição
	Os serviços ofertados pela empresa tambem possuem um custo de manutenção, em geral, quantidade de dados transferidos e armazenados, e quantidade de requisições a API. Custo esse que é repassado para os contratantes do serviço.

	% contexto
	O backend como serviço, assim como todo software que roda como serviço na nuvem, ele é mantido por uma empresa que obviamente vai cobrar pelo serviço, por tanto, deve haver uma preocupação com o valor a ser pago para o fornecedor dessa solução.
	
	O preço varia muito de acordo com a quantidade do uso do serviço, que depende do conjunto de varios fatores, dentre ele podemos citar, por exemplo no firebase estes:
		ponto 1 paga por quantidade de dados consumidos
		ponto 2 paga pela quantidade de usuarios simultaneos
		...

	% falar dos planos que o firebase oferece
	% nao esquecer a referencia do site

\section{Desafios}
\label{sec:challenges}
	% não dar ideia de bug... nao vai gerar problemas, vai gerar desafios.. questoes de projetos que levam aos desafios da escolha
	
	% contexto
	-> ter um servidor escalavel para aguentar todos os usuarios de forma simultanea
	-> trocar o provedor do servico é custoso
	-> pagar menos pelo serviço contratado
 	-> ter menor tempo de resposta do servidor

	% definição
	-> definir queries que minimizem a quantidade de acessos/requisições a API do servidor
	-> minimizar banda transferida entre cliente e servidor, alem de ser bom para o cliente, tambem minimiza o valor final da conta no final do mes.
	% algum outro desafio?


	\subsection{Escalabilidade}%7:30
	\label{subsec:scalability}
	% contexto
	Para oferecer um serviço que seja escalavel e que estaja disponivel na maior parte do tempo para o cliente, é necessario um grande investimento em infraestrutura por parte dos BaaS e é importante ter em mente que para cada fornecedor existem maximos e minimos com relação a estrutura que eles fornecem. 

	O Firebase consegue lidar com centenas de milhares de usuarios conectados ao mesmo tempo tanto para acesso ao banco de dados como para o armazenamento de arquivos, mas caso um dos nós passa a ter mais de 1--10+ milhões de filhos (digamos que você estaja escrevendo um app de mensagens e colocou todas as mensagens de um certo periodo em um mesmo ramo `\textit{/root/messages}') o sistema passa a ter um decaimento da performance. A razão para essa limitação é que o Firebase é armazenado em memoria ``quente'' então quando se tenta acessar um nó, todos os filhos são baixados e isso acaba lotando a memoria causando uma queda no desempenho. Uma solução para esse problema seria estruturar seus dados de forma que impeça isso de acontecer (por exemplo, arquivando chats que são mais antigos que 1 dia, uma semana, um mes). 	

	Outros fatores que devem ser considerados para se ter um sistema escalavel no Firebase são os padrões de acesso. Digamos que para gerar a lista de usuarios online desse app de mensagens, voce pode iterar sobre toda a lista de usuarios verificando se por um atributo para decidir se o usuario está online, ou voce desnormalizar os dados e ter um nó separado `online-users'. Ao utilizar a segunda abordagem, a quantidade de informações trafegada visto que você está pegando exatamente o que voce quer ao invez de derivar ela de outros dados. Um contra dessa abordagem é que ela requer mais espaço no banco. Um dos lemas do Firebase é que espaço é barato, o tempo do usuario não então cabe ao desenvedor decidir qual tipo de abordagem seguir. veja \ref{subsec:min_space}, \ref{subsec:min_trafic}
	%\cite{https://groups.google.com/forum/#!topic/firebase-angular/OY7xtpoc9XA}
	
	%voce deve pagar mais por um servico que seja de certa forma mais escalavel, como abordadar a precificação de um servico "mais" escalavel? 

	\subsection{Flexibilidade}%8:55 {antigo 'mudança de provedor'}
	\label{subsec:flexibility}

	% contexto
	dentro da manutenção de software uma das ações que podem ser realizada é a troca de fornecedor de algum serviço, por exemplo a troca do backend como serviço utilizado.

	dependendo da estrutura utilizada para armazenar os dados no serviço, trocar de provedor pode ser uma tarefa complicada, pois nao necessariamente os dados serao armazenados da mesma forma, algumas APIs disponibilizadas por um podem nao ter a mesma funcionalidades que o do outro... toda a parte que trata da comunicação entre o servidor e a criação dos objetos tera de ser repensada.

	% definição
	

	% soluções utilizadas

	\subsection{Segurança}%9:30
	\label{subsec:security}
	% contexto

	% definição
	um backend deve oferecer segurança no serviço que ele oferece, seja ele um serviço ou gerido por um servidor ques esta instalado num predio da organização.

	os baas em geral tratam daquela maneira
		utilizam um canal seguro via tts para transmissao de informaçoes
		e portanto o cliente deve estar praparado para esse tipo de comunicação

	ponto importante a destacar
	em relação aos dados em geral não temos muita informação, isso deve ser implementado pelo desenvolvedor. portanto boa parte da logica de segurança fica implementada na aplicação cliente. deve se ter um cuidado de como é feita a implementação para não expor a todo o sistema pois o codigo está no cliente.

	% soluções utilizadas

	\subsection{Sincronização de dados}%14:40
	\label{subsec:data_sync}
	% tratar garantia de integridade junto
	% contexto
	
	% definição
	é uma ação que é muito importe entre o sistema que trata de diferentes versoes da informaçao em diferentes clientes. esse tipo de atividade geralmente gera um problema arquitetural que envolve tanto o momento que vai ser sincronizado e o que vai ser sincronizado e como vai ser sincronizado.

	alguns serviços fazem uma camada de transparencia...
	trafegam apenas as mudanças.

	% soluções utilizadas
	 - Firebase Realtime Database \ref{subsec:data_management}


	\subsection{Garantia da Integridade}%15:50
	% contexto

	% definição

	% soluções utilizadas

	\subsection{Garantia da consistencia da informação}%16:30
	% contexto

	% definição

	% soluções utilizadas

	\subsection{Minimizar trafego}
	\label{subsec:min_trafic}
	% contexto
	Um dos fatores que influencia nas taxas pagas para o fornecedor do BaaS é a quantidade de dados trafegados. Quando mais dados trafegados, maior será a quantia a ser paga no final do mês.

	% definição
	- quanto mais dados trafegados, maior a quantia paga

	- firebase tem distinção entre storage e database

	% soluções utilizadas
	- denormalização de dados

	- contratar mais banda a medida que necessario

	\subsection{Minimizar espaço utilizado}
	\label{subsec:min_space}
	% contexto
	BaaS em geral oferecem uma cota de espaço em disco limitada pelo serviço contratado, aonde se tem limites de tranferencia alem de limites de armazenamento.
	
	% definição
	armazenamento de arquivos e trafego dos mesmos impacta direto na quantia a ser paga no final do mes. assim o principal desafio é tentar minimizar o tamanho de arquivos trafegados usando soluções como compressão desses mesmos ou até mesmo usar outra solição BaaS para armazenamento de dados.

	% soluções utilizadas
	- compactação de arquivos antes de enviar nos disposivos clientes.

	- usar outro serviço somente para armazenamento de arquivos.

	\subsection{Recuperação de dados}%19:30
	% contexto

	% definição
	
	% soluções utilizadas
