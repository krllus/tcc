\chapter{Fundamentação Teórica}
\section{Cloud Computing}
	% definition
	Cloud Computing ou computação na nuvem, é um termo que pode ser usado para descrever uma plataforma que provê, configura, reconfigura e desprovê servidores a medida do necessário. Esses servidores podem ser maquinas fisicas ou maquinas virtuais e podem  ser utilizados para descrever aplicações que são extendidas para serem acessadas através da internet. Elas são providas por data centers, ou centro de dados, que hospedam web services e WebApplications, tornando possível qualquer um com um navegador web ser capaz de acessar essas aplicações na nuvem.
	
	No mercado, alguns famosos provedores de computação na nuvem são a Amazon EC2 \ref{X} e IBM Bluemix \ref{X}.
	
	https://antonioricardo.org/2013/03/28/o-que-e-saas-iaas-e-paas-em-cloud-computing-conceitos-basicos/

	\subsection{Os serviços de software na nuvem}

	\subsection{Tipos de Provedores} 
		Provedores de serviços em nuvem podem ser separados em categorias, todas seguindo um padrão quanto ao modelo de pagamento, porem com suas devidas peculiaridades.
	
		\begin{description}
		\item [Software como serviço - SaaS]
		Clientes de software como serviço alugam o uso de aplicações que rodam dentro da infraestrutura do provedor do serviço de computaço na nuvem. Essas aplicações são geralmente ofertadas para os clientes atraves da internet e são gerenciadas totalmente pelo provedor do serviço. O maior beneficio de SaaS é a manutenção de uma mesma versão do software para todos os cliente. Novas funcionalidades podem ser integradas pelo provedor através de uma simples atualização centralizada de arquivos e assim disponivel para todos os clientes.

		Ao contrario do modelo tradicional de software comprado por licença para utilização do produto, em SaaSs, a precificação geralmente é cobrada com base em assinaturas de conjunto de serviços.

		% tipos de SaaS

		% example
		Skype, Microsoft Word, 

		\item [Infraestrutura como serviço - IaaS]
		Infraestrutura como serviço (Infrastructure as a Service - IaaS) entrega recursos de hardware tais como processador, espaço de disco ou componentes de rede como um serviço. Esses recursos são geralmente entregues como uma plataforma virtual pelo provedor do serviço e pode ser acessada atraves da Web pelo cliente. O cliente tem total controle sobre a maquina virtual e não é responsavel por gerenciar a infraestrutura que mantem essa maquina.
	
		% Como é feita a cobrança
		A cobrança nos IaaS é realizada de acordo com fatores como o numero de servidores virtuais, quantidade de dados trafegados, dados armazenados, horas de funcionamento do servico e outros fatores que variam entre provedores.

		%example
		Amazon EC2
	

		\item[Plataforma como serviço - PaaS]
		% fix english translation, last sentence.
		Plataforma como serviço (Plataform as a Service - PaaS), é um modelo intermediario entre IaaS e SaaS, aonde o desenvolvedor tem toda as vantagem de ter seu software em nuvem, junto com a vantagem de não precisar de ter uma infraestrutura local. Os provedores de PaaS oferecem a plataforma de uma aplicação como serviço. Assim os clientes podem disponibilizar software usando as ferramentas e linguagens de programação disponibilizadas pelo provedor e a ter controle sobre suas aplicações via um painel de controle.

		% escrever mais sobre este conceito para que tenhamos certeza da diferença dele com o SaaS.
		% example
		Microsoft Azure
		Google App Engine
		Amazon EC2
		% nao acho que amazon ec2 entre como PaaS, o painel deles parece que controla apenas os servidores e nao dados da aplicação tais como numeor de usuarios ativos, crashes... posso estar enganado, mas pelo pouco que li no site me parece que é isso mesmo.
		IBM Bluemix
		Heroku

		\item[Armazenamento como serviço - STaaS]
		Armazenamento como Serviço (Storage as a Service - STaaS) é um modelo de negocio onde o fornecedor do serviço aluga parte da sua infraestrutura de armazenamento de dados para o cliente e é mantido com base em uma assinatura. Pelo provedor ter uma estrutura escalavel, ele é capaz de ofertar armazenamento muito mais compensador que a maioria dos individuos ou corporações podem fornecer com seu proprio armazenamento, principalmente se levarmos em conta os custos de pessoas e manutenção de equipamentos. STaaS são geralmente usados quando se quer fazer um backup não local. Um dos pontos fracos desse serviço: é necessario uma grande quantidade de largura de banda para utilizar armazenamento em nuvem.
	
		%example
		Dropbox, Github 

		\item[Segurança como serviço - SECaaS]
		Segurança como um serviço (Security as a Service - SECaaS) é um modelo de negocio onde provedores integram seus serviços a infraestrutura de terceiros, cujo objetivo é adicionar funcionalidades de segurança ao software em si, como firewall, verificação de anti-vírus ou tráfego de informação criptografada.
	
		% example - Gmail mcCafee 
		Aion Cloud, Cloudbric, 

		\item[Dados como serviço - DaaS]
		% nao entendi o final desse paragrafo.
		% o que seria UN?
		Dados como serviço (Data as a Service - DaaS) é baseado no conceito de que o produto (dados) pode ser providenciado sob demanda pelo usuario, não importando localização geografica ou separação organizacional entre provedor e consumidor. DaaS providos como serviço foram usados no começo somente em ?web mashups?, porem agora seu uso esta aumentando comercialmente e, menos comum, em organizações tais como a UN (EVITAR SIGLAS) \ref{X}.
		%example
	
		%Verificar para talvez refazermos ou até exlcuirmos este tópico.

		\item[Ambiente de testes como serviço - TEaaS]
		Ambiente de testes como serviço (Test Environment as a services - TEaaS), geralmente referenciado como "ambiente de teste sob demanda", é um ambiente distribuido de testes em que software e os dados associados são hospedados e executados na nuvem afim de realizar verificações, validações ou o teste de software em si. Em geral, este tipo de software é utilizado por equipes que no podem ter acesso a dispositivos ou contextos específicos do usuário final, incluindo testes de escalabilidade.
	
		%example
		Firebase Test Lab

		\item[Backend as a Service - BaaS]
		Tambem conhecido como Mobile Backend as a Service (MBaaS), é um modelo que provê para desenvolvedores web e móvel, uma maneira de ligar suas aplicações a uma plataforma de backend e armazenamento na nuvem, alem de prover funcionalidades como gestão de usuarios, notificações e integração com serviços de redes sociais. Esses serviços são providos via o uso de Software Development Kits (SDKs) e Application Programming Interfaces (APIs). 
	
		O conceito de BaaS é relativamente recente no mundo de computação em nuvem\ref{API_Evangelist_Overview_of_the_Backend_as_a_Service_Space}, sendo que no mercado, a maioria das startups de BaaS são datadas do começo de 2011 ou depois.

		MBaaS vem da frustração em implantar nas plataformas de IaaS, e de que PaaS não oferecem o que é necessario para plataformas moveis. BaaS é sobre abstrair as complexidades de lançamento e gerenciamento da sua propria infraestrutura e trazer recursos que visão(targeting) exatamente o que desenvovedores precisam para contruir as proximas gerações de aplicativos moveis.\ref{API_Evangelist_Overview_of_the_Backend_as_a_Service_Space}

		BaaS tem a mesma intenções do PaaS: acelerar o processo de desenvimento. Porem, BaaS se concentram em trazer um infraestrutura que escala automaticamente e otimiza uma serie de recursos essenciais que desenvovedores precisam, tais como ferramentas de conteudo, dados, mensagens e APIs de terceiros tais como Facebook, Twitter e Dropbox.


		%example - lembrar do arquivo que tem as listas e do PFC da Francielly
		Appcelerator, Appery.io, Firebase, Kinvey

		\end{description}

	% não está certo aonde deveria ficar essa parte visto que ela é de BaaS quase que exclusiva do artigo do API Evangelist
	% talvez separar BaaS como uma subsecao e esses topicos entratem dentro como sento subsubsecao...

	\subsection{quais são os produtos do mercado?}
	%citação trabalho da francielly

	\subsection{Quais são os Building Blocks entre BaaS}
	% building block - elementos base
	Atualmente existem varios provedores de BaaS, \cite{subsection_acima} cada um ofertando um leque de diversas funcionalidades. Analizando esse conjunto, é possivel extrair quais são as funcinalidades base do qual um provedor SaaS é composto.

	% deixar \item[] no comeco para pular para proxima linha 
	\begin{description}
		\item[Gerenciamento de Usuários]
			\begin{description}
            	\item[]
				\item[Users]{Gerenciamento de usuarios é prover ao  desenvolvedor meios para o qual ele possa permitir que o usuario entre no aplicativo e comece a interagir com a aplicação. É uma funcinalidade principal de todos os BaaS.}
			\end{description}
		
		\item[Sistema de Gerenciamento de Conteúdo]
			\begin{description}
            	\item[]
				\item[Basic]{Sistema basico de gerenciamento de conteúdo que permite a criação, edição e deleção de paginas de conteudo generico. Em geral essa funcionalidade se asemelha com gerenciadores de conteúdo (CMS) da Web, porem aplicados de maneira simples para aplicações mobile.}
			\end{description}
		
		\item[Dados]
			\begin{description}
            	\item[]
				\item[Data Browser] { Prover uma maneira rapida e alternativa de navegar entre dados que sao usados no backend da aplicacao. Navegadores de dados variam de implementaçṍes bem simples até umas mais complexas e com interfaces ricas.}
				
				\item[Key Value] {Armazenamento do tipo chave-valor permite aplicações armazenar dados de forma não relacional, assim podendo ser mantida em objetos ou estruras de dados da linguagem utilizada para programar.}
				
				\item[MySQL] {MySQL é um banco de dados popular para construir aplicações web.  BaaS oferecem conectores MySQL para que eles sejam usados como fonte de dados para aplicações mobile.}
				
				\item[Relational] { Tabelas relacionais permitem conectar uma tabela de dados em diferentes tabelas criando um relacionamento entre os campos e valores. Em geral, a habilidade de criar relações entre dados é distribuida junto com conector de MySQL. Porém, de forma mais simples, fornece-se o basico para que os desenvolvedores possam desenvolver suas aplicações de maneira mais rapida.}

				\item[Table] { Permite que desenvolvedores armazenem seus dados em forma de linhas, colunas e tabelas. Geralmente requisito basico para provedores de BaaS.}

			\end{description}
		
		\item[Imagens e Fotos]
			\begin{description}
            	\item [] 
				\item[Storage] { Permite gerenciamento de imagens e fotos.}
			\end{description}
		

		\item[Custom]
			\begin{description}
            	\item[]
				\item[Code] { Alem de permitir desenvolvedores a criar suas aplicações em menos tempo, os frameworks de BaaS, tambem permitem a injeção de codigo em areas controladas da plataforma de maneira a alterar o comportamento base de funcionalidade de forma a melhor atender as necessidades do cliente.}

				\item[Objects] { Permite ao desenvolvedor criar sua propria estrutura de dados, sendo possivel armazenar qualquer tipo de dados desejado pelo desenvolvedor. }
			\end{description}
	

		\item[API]
			\begin{description}
            	\item[]
				\item[Custom] { Varios provedores de BaaS fornecem ferramentas que permitem ao desenvolvedor criar suas proprias integrações com a plataforma. Essa é uma das maneiras mais rapidas de se extender as funcionalidades basicas oferecidas pelo provedor. }

				\item[Query] { Muitos provedores, oferecem interfaces de busca, no qual buscas que seguem o estilo SQL são utilizadas para recuperar dados armazenados. }

				\item[REST API] { Todas as plataformas de BaaS fornecem APIs, que abrangem todas as funcionalidades da plataforma, desde gerenciamento de usuarios até o armazenamento de dados. São usadas tanto na construção de aplicações como na integração com outros sistemas de gerenciamento. }
			\end{description}
		
		\item[Comercio Virtual]
			\begin{description}
            	\item[]
				\item[In-App Purchases] { No modelo atual, é necessario a existencia de alguma forma de comercio dentro das aplicações. A maneira mais comum de fazer isso é via compras dentro do aplicativo. Provedores dão a possibilidade de o desenvolvedor implementar compras usando moedas virtual dentro de seus aplicativos. Compras dentro de aplicativos é uma forma de gamificar o app e deixa-lo mais apelativo para o usuario.}
			\end{description}
		
		\item[Monetização]
			\begin{description}
            	\item[]
				\item[Promotions] { Para apoiar estrategias de monetização, incentivo e fidelidade, os provedores estão contruindo ferramentas de promoção que permitem aos desenvoleores, a capacidade de promover seus produtos entre seus usuarios.}
			\end{description}
		
		\item[Ranking]
			\begin{description}
            	\item[]
				\item[Ratings] { Dar ao usuario a capacidade de classificar quaquer coisa, está se tornando comum entre os provedores, assim, plataformas BaaS estão oferecendo aos desenvolvedores, ferramentas basicas para classificação de coisas como parte de seu conjunto basico de ferramentas.}
			\end{description}
		
		\item[Comunicação]
			\begin{description}
				\item[]
                \item[Chat] { Comunicação em tempo real via chat é uma tarefa comum entre aplicações web e moveis. BaaS tendem a utilizar API's de terceiros para prover essa funcionalidades.}

				\item[Email] { A manutenção e configuração da infraestrutura de um servidor de email, consome muito tempo dos desenvolvedores, fazendo com que BaaS forneçam uma implementação desse serviço que pode ser propria ou advinda de um provedor de email maior.}

				\item[Messaging System] { Assim como chat, SMS e email, provedores BaaS fornecem ferramentas completas de envio e gerenciamento de mensagens. Não ter que criar seu proprio sistema é de gerenciamento de mensagens é uma verdadeira mão na roda para desenvolvedores.}
				
				\item[Push Notifications] { Notificações Push são comuns entre usuarios de smartphones. Usuarios esperam que a informação chegue até eles. Dentre a categoria de comunicação e troca de mensagens, notificações push é a ferramenta mais comum entre os provedores de BaaS.}

				% nao concordo com o que está escrito no texto do API Evangelist 
				\item[SMS] { No mundo de aplicações moveis, a comunição via SMS é dominante, e uma escolha comum entre os desenvolvedores para mandar e recever mensagens.}
			\end{description}
		
		\item[Geo]
			\begin{description}
				\item[]
                \item[Location] { Providencia aos desenvolvedores, uma forma nativa de identificar a localização de usuarios que estão usando uma plataforma movel ou conectado a um computador com internet. Prove a localização de um usuario como sendo um ponto, latitude longitude ou um endereço}
				
				\item[Places] { Provê aos desenvedores ferramentas que frnecem informações precisas sobre locais e pontos de interesse para o usuario. BaaS fornecem essas ferramentas atraves de parceiro(3rd partner)}
				
				\item[Targeting] { ter os usuarios como alvo, targeting, geralmente feito com base em varios pontos incluindo localização, comportamento, historico, amigos entre outros. Aproximando do usuario exatamente no seus pontos de interesse. Esse tipo de abordagem demanda tempo e esforço, para ser contruido, BaaS estão }
			\end{description}
		
		\item[Jogos]
			\begin{description}
				\item[]
                \item[Focus] { Capacitar o desenvolvimento de jogos para plataformas moveis e web, é comum entre provedores de BaaS. Jogos (gaming) é de longe o maior foco de provedores de BaaS.}
			\end{description}

		\item[Availability]
			\begin{description}
				\item[]
                \item[Scaling] { Abstrair a complexidade da infraestrutura é uma principais promessas dos BaaS. Prover uma plataforma que se escalavel automaticamente para os desenvolvedores a medida do necessario é o motivo numero 1 para a utilização de BaaS.}
			\end{description}

		\item[Dispositivos Moveis]
			\begin{description}
				\item[]
                \item[Android] { Android é a plataforma numero dois em que os desenvolvedores estão desenvolvendo para. A maioria dos provedores de BaaS fornecem suporte para o desenvolvimento aplicações Android nativas. }

				\item[Blackberry] { Blackberry é ainda utilizado no mundo de Business e alguns nichos ao redor do mundo. Alguns provedores de BaaS fornecem SDK's e outras ferramentas especificas para se trabalhar com a plataforma. }
				
				\item[iOS] { iOS é a plataforma numero um em que os desenvolvedores estão desenvolvendo para. Todos os provedores de BaaS fornecem ferramentas para o desenvolvimento de aplicações para iPhone e iPad. iOS é a motivo da existencia de muitos BaaS. }
				
				\item[Windows] { Windows Phone esta trabalhando duro para ter uma fatia do mercado de smartphones. A maioria dos provedores de BaaS dominantes, provem SDK's especificas e outras ferramentas para a Windows Phone.}
			\end{description}

		\item[Mobile Platforms]
			\begin{description}
				\item[]
                \item[PhoneGap] { Phonegap é um framework de desenvolvimento movel produzido pela Nitobi que foi comprada pela Adobe Systems. Ele permite a contrução de aplicativos usando JavaScript, HTML5 e CSS3. Phonegap é conhecido pelo desenvolvimento de aplicações hibridas que estão usando tecnologias web, mas que tem acesso a uma das melhores ferramentas do desenvolvimento nativo para iOS e Android. Muitos provedores BaaS usam phonegap como a tecnologia core para o desenvolvimento de aplicaçoes moveis. }

				\item[Titanium] { Appcelerator Titanium é uma palataforma para o desenvolviento de aplicativos para smartphones, tablets e desktops que usa tecnologias web providas por Appcelerator Inc. Com suporte para iOS, Android e Blabkberry. Apesar de Appcelerator prover sua propria plataform de BaaS, Titanium é fornecido como parte das ferramentas de outros BaaS.}
			\end{description}

		\item[Linguagens]
			\begin{description}
				\item[]
                \item[HTML5] { HTML5 é uma linguagem de marcação usada para o estruturamento e apresentação de conteudo para a World Wide Web e é uma das tecnologias cores da Internet. Essa linguagem se tornou padrão para o desenvolvimento de aplicativos moveis não nativos ou web-based. Muito dos BaaS fornecem html5 como alternativa ao desenvolvimento de aplicações nativas Android e iOS.}

				\item[PHP] { PHP é a linguagem top entre as usadas para programação web. O objetivo dos BaaS é fazer o desenvolvimento de aplicativos, principalmente os moveis, mais facil e que abranja uma maior audiencia. Com isso em mente, BaaS dão suporte aos desenvolvedores PHP para fazer essa transição de web para mobile.}
			\end{description}

		\item[Social]
			\begin{description}
				\item[]
                \item[Facebook] { Facebook é a cereja do bolo no ramo das redes sociais. É uma plataforma que está incorporada em varios provedores de BaaS, permitindo os desenvolvedores a facilitar a autenticação, criação de posts, compartilhamento e outras tarefas comuns de redes sociais para seus usuarios. }

				\item[Twitter] { Twitter é a cereja do bolo de micro blogging. Assim como no facebook, la está incorporada em varios provedores de BaaS e fornece para os desenvolvedores, ferramentas para permitir que os seus usuarios se autentiquem, criem e compartilhem conteudos de maneira mais facil. }
			\end{description}

		\item[Enterprise]
			\begin{description}
				\item[] 
                \item[Support] { Muitos provedores de BaaS, fornecem soluções para empresas, assim sendo necessario prover como um de seus serviço, o suporte para empresas.}
			\end{description}
	\end{description}

	\subsection{Preficicação}
	% abordar precificação do firebase aqui
	O modelo de precificação para provedores de BaaS ainda esta evoluindo. Muito dos provedores, adotou o custo de armazenamento e chamadas de API como sendo o seu modelo, porem ainda existe muito o que debater sobre o que é modelo mais proveitoso para os desenvolvedores e que gera lucro para o provedor.

	Atualmente alguns dos provedores de BaaS operam usando serviços de terceiros, tal como Amazon Web Services, o que faz com que eles cobrem pelos custos da plataforma. Os modelos mais utilizados para cobrança são chamadas de API e Armazenamento.

	\begin{description}
		\item[]
		\item[API Calls]{ Cobrar do desenvolvedor pela quantidade de chamadas a APIs. O valor de um BaaS está nos recursos que as APIs fornecem, o modelo de negocio é construido sob o valor dessas APIs.}
		
		\item[Storage]{ Armazenamento é o maior custo para os provedores de BaaS. Para os desenvolvedores, é comum o armazenamento de dados, arquivos e objetos usando o storage do provedor. Grande parte dos provedores BaaS rodam em nuvem, assim esse armazenamento é sempre uma solução em nuvem. }
	\end{description}

	Apesar de esses serem os metodos mais comuns para cobrança, existem outras varias requisitos por onde o provedor de BaaS pode cobrar.


	O foco de varios BaaS é o mundo de aplicativos moveis, assim, faz sentido cobrar dos desenvolvedores por quantidade de usuarios finais.	Alguns provedores dizem que cobrar pelo numero de chamadas de API e pela quantidade de usuarios ativos, é uma forma de medir o sucesso do aplicativo, se existem varios usuarios e varias chamadas da API, significa que o aplicativo é um sucesso, e assim o desenvolvedor está apto a pagar um pouco mais pelas serviços do provedor.
	
\section{Banco de dados como Serviço: Questões técnicas e soluções}
