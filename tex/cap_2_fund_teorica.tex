\chapter{Fundamentação Teórica}
\section{Cloud Computing}
% definition
	Cloud Computing ou computação na nuvem, é um termo que pode ser usado para descrever uma plataforma que provê, configura, reconfigura e desprovê servidores a medida do necessario. Servidores esses que podem ser maquinas fisicas ou maquinas virtuais. Tambem pode usado para descrever aplicações que são extendidas para serem acessadas através da internet. Aplicações essas que são providas por Datacenters, ou centro de dados, que hospedam WebServices e WebApplications, tornando possível qualquer um com um navegador web ser capaz de acessar essas aplicações na nuvem.

\subsection{Os serviços de software na nuvem}
		
% providers types
\subsection{Software as a Service - SaaS}
	Clientes de software como serviço, alugam o uso de aplicações que rodam dentro da infraestrutura do provedor do serviço de Cloud Computing. Essas aplicações são geralmente ofertadas para os clientes atravez da internet e são gerenciadas totalmente pelo provedor do serviço. O maior beneficio de SaaS é que todos os clientes estão usando a mesma versão do software e novas funcionalidades podem ser facilmente integradas pelo provedor e assim disponiveis para todos os clientes.

	% example

\subsection{Platform as a Service - PaaS}
	% fix english translation, last sentence.
	Provedores de PaaS oferecem a plataforma de uma aplicação como serviço. Isso permite clientes a disponibilizar softwares usando as ferramentas e linguagens de programação disponibilizadas pelo provedor e a ter controle sobre as aplicações e environment-related settings.

	% example
	Google App Engine

\subsetcion{Infrastructure as a Service - IaaS}
	Infraestrutura como serviço entrega recursos de hardware tais como CPU, espaço de disco ou componentes de rede como um serviço. Esses recursos sao geralmente entregues como uma plataforma virtual pelo provedor do serviço e pode ser acessada atravez da Web pelo cliente. O cliente tem total controle sobre a maquina virtual e não é responsavel por gerenciar a infraestrutura por que mantem essa maquina.
	%example	

\subsection{Storage as a Service - STaaS}
	Storage as a Service é um modelo de negocio onde o fornecedor do serviço aluga parte da sua infraestrutura de armazenamento de dados para o cliente e é mantido com base em uma assinatura. Por o provedor ter uma estrutura escalavel, eles são capazes de ofertar armazenamento muito mais custo-efetivo que a maioria dos individuos ou corporações podem fornecer seu proprio armazenamento, quando custo total de propriedade é considerado. STaaS são geralmente usados quando se quer fazer um backup não local. Um dos pontos fracos desse serviço: é necessario uma grande quantidade de largura de banda para utilizar armazenamento em nuvem.
	
	%example
	Dropbox
\subsection{Security as a Service - SECaaS}
	% nao entendi muito bem...
	Segurança como um serviço, é um modelo de negocio onde provedores integram seus serviços a infraestrutura de uma empresa
	%example

\subsection{Data as a Service - DaaS}
	% nao entendi o final desse paragrafo.
	% o que seria UN?
	Dados como serviço, é baseado no conceito de que o produto, dados nesse caso, pode ser providenciado sob demanda pelo usuario não importando localização geografica ou separação organizacional entre provedor e consumidor. DaaS providos como serviço foi no começo usando somente em web mashups, porem agora esta seu uso esta aumentando, comercialmente e, menos comum, em organizações tais como a UN.
	%example

\subsection{Test Environment as a Service - TEaaS}
	Ambiente de testes como serviço, geralmente referenciado como "ambienete de teste sob demanda", é um ambiente distribuido de testes em que software e os dados associados sao hospedados, geralmente em nuvem, e são acessados pelos usuarios atraves de navegadores web.
	%example

\subsection{Backend as a Service}
	Tambem conhecido como Mobile Backend as a Service, Backend Movel como serviço, é um modelo que provê para desenvolvedores web e mobile, uma maneira de ligar/linkar suas aplicações a uma plataforma de backend e storage na nuvem, alem de prover funcionalidades tais como manuzeamento de usuarios, notificações push, e integração com serviços de redes sociais. Esses serviços são providos via o uso de Software Development Kits (SDKs) e Application Programming Interfaces (APIs). BaaS é relativamente recente no mundo de computação em nuvem, a maioria das startups de BaaS são datadas do começo de 2011 ou depois.
	%example
	Firebase, Parse, Heroku

	\subsection{quais são os building blocks?}
	\subsection{quais são os produtos do mercado?}
		citação trabalho da francielly
	\subsection{preficicação (mostrar do Firebase)}



\section{Banco de dados como Serviço: Questões técnicas e soluções}
